\documentclass{article}
\usepackage{graphicx} % Required for inserting images

\title{Discrete Math}
\author{andriel vinicius}
\date{\today}

\begin{document}

\maketitle

\section{Introdução}
\par
Diferente das outras ciências, na matemática não é a \emph{experiência} que prova seus fatos, mas sim a \textbf{lógica e dedução}.
Uma teoria é apresentada por meio de \emph{provas/demonstrações}, e a validação se dá a partir da aceitação de outros matemáticos. Cabe a nós, portanto, se perguntar: o que danado a matemática estuda?
\par
A resposta a essa pergunta pode ser intrigante: \textit{a matemática estuda os elementos \textbf{abstratos} da natureza}. Que objetos matemáticos são esses? Qualquer coisa; não importa com o que se esteja trabalhando na vida real, quando traduzido para o campo da matemática, todos os elementos se tornam ideias e o seu estudo e análise se dá por meio do \emph{raciocínio lógico}.
\par
Devido à abstração ao mundo das ideias, exige-se, na matemática, uma linguagem \emph{formal} e focada na \emph{lógica} na escrita de suas proposições, visto que, para que algo seja provado, sua justificativa tem de ser bem elaborada e que faça sentido lógico! Veja alguns casos abaixo:
\subsection{Exemplo A}
Tome a equação $\frac{2}{b} = \frac{2}{7} $ como verdadeira. Podemos simplesmente falar que $ b = 7 $ pois os numeradores são iguais? \emph{Não}! Para chegar nessa conclusão, devemos partir da equação modelo $ \frac{a}{b} = \frac{a}{c} $; elas só podem ser iguais se $a = 0$ \emph{ou} $b = c$. Voltando ao problema real, dado que $ a \ne 0 $, então, se $\frac{2}{b} = \frac{2}{7}$ é verdadeiro, então $b = 7$!
\subsection{Exemplo B}
Tome a equação $ \frac{x-1}{x} = \frac{x - 1}{ x + 1} $ como verdade. Perceba que, como no caso anterior, a equação só é verdade se $x - 1 = 0$ \emph{ou} $x = x + 1$. Note, contudo, que a segunda condição é um \emph{absurdo}, pois $x$ não é igual a $x + 1$. Portanto, temos a primeira condição a analisar:
\begin{math}
x - 1 = 0
\mapsto x = 0 + 1
\mapsto x = 1
\end{math}
Assim, está provado que "por a + b" que $ x = 1 $!

\subsection{Demonstrações}
Essas operações que acabamos de fazer são denominadas \textbf{demonstrações}. É por meio dessa comunicação que as ideias do matemático são registradas e passadas adiante. Elas são feitas seguindo os princípios da \emph{lógica matemática}.

\subsection{Lógica matemática}
Oriunda da filosofia clássica - lógica essa que trata das formas do pensamento em geral de modo a determinar o que é verdadeiro ou falso -, foi desenvolvida principalmente a partir do século XIX a partir da contribuição de \emph{filósofos} e \emph{matemáticos}.
\par
Para a lógica matemática foi criada uma espécie de \textbf{língua própria} para expressar as ideias dessa lógica, de modo a \emph{uniformizar} a escrita das operações lógicas e suprimir eventuais paradoxos e imprecisões linguísticas.
\subsubsection{Sistema axiomático}
Sistema no qual toda a lógica matemática foi desenvolvida, baseada primariamente nos chamados \textbf{axiomas}, afirmações aceitas como um ponto de partida que têm como fim chegar em um ponto específico.
\par
Esse sistema foi proposto por um matemático de nome "Hilbert" em 1900, que propôs que criassem um sistema axiomático do qual \emph{toda matemática poderia derivar}.
\par
Tal desafio resultou em contribuições fundamentais para áreas específicas da matemática, especialmente na \emph{Álgebra} e na \emph{Teoria dos Conjuntos}, essa que teve muita contribuição do grande Georg \textbf{Cantor}.
\par
Contudo, provou-se que, se um sistema axiomático for rico o suficiente para construir a matemática, ele:
\begin{enumerate}
    \item não será \textbf{completo}, pois há verdades que não podem ser provadas;
    \item não pode provar que não tem contradições em si mesmo.
\end{enumerate}
\par
Mas o que se tem hoje em dia? Bem, os matemáticos utilizam o sistema axiomático denominado \emph{ZFC}, que não vou adentrar aqui pois não nos interessa, é apenas a título de curiosidade.

\subsection{Matemática Discreta}
"Tá, mas onde você quis chegar com toda essa introdução e contextualização? Cadê a matemática discreta?". Bem, aqui estamos: a matemática discreta é uma área da matemática que trata acerca das \textbf{estruturas matemáticas que são finitas e/ou podem ser enumeradas}; um bom exemplo disso são os números naturais. Ela apresenta métodos de resolução diversos e utiliza constantemente do \emph{raciocínio matemático}.

\section{To be Continued ...}

\end{document}
