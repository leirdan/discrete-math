\documentclass{article}
\usepackage{graphicx} % Required for inserting images
\usepackage{amsmath}
\usepackage{array}

\title{Discrete Math}
\author{andriel vinicius}
\date{\today}

\begin{document}

\maketitle

\section{Introdução}
\par
Diferente das outras ciências, na matemática não é a \emph{experiência} que prova seus fatos, mas sim a \textbf{lógica e dedução}.
Uma teoria é apresentada por meio de \emph{provas/demonstrações}, e a validação se dá a partir da aceitação de outros matemáticos. Cabe a nós, portanto, se perguntar: o que danado a matemática estuda?
\par
A resposta a essa pergunta pode ser intrigante: \textit{a matemática estuda os elementos \textbf{abstratos} da natureza}. Que objetos matemáticos são esses? Qualquer coisa; não importa com o que se esteja trabalhando na vida real, quando traduzido para o campo da matemática, todos os elementos se tornam ideias e o seu estudo e análise se dá por meio do \emph{raciocínio lógico}.
\par
Devido à abstração ao mundo das ideias, exige-se, na matemática, uma linguagem \emph{formal} e focada na \emph{lógica} na escrita de suas proposições, visto que, para que algo seja provado, sua justificativa tem de ser bem elaborada e que faça sentido lógico! Veja alguns casos abaixo:
\subsection{Exemplo A}
Tome a equação $\frac{2}{b} = \frac{2}{7} $ como verdadeira. Podemos simplesmente falar que $ b = 7 $ pois os numeradores são iguais? \emph{Não}! Para chegar nessa conclusão, devemos partir da equação modelo $ \frac{a}{b} = \frac{a}{c} $; elas só podem ser iguais se $a = 0$ \emph{ou} $b = c$. Voltando ao problema real, dado que $ a \ne 0 $, então, se $\frac{2}{b} = \frac{2}{7}$ é verdadeiro, então $b = 7$!
\subsection{Exemplo B}
Tome a equação $ \frac{x-1}{x} = \frac{x - 1}{ x + 1} $ como verdade. Perceba que, como no caso anterior, a equação só é verdade se $x - 1 = 0$ \emph{ou} $x = x + 1$. Note, contudo, que a segunda condição é um \emph{absurdo}, pois $x$ não é igual a $x + 1$. Portanto, temos a primeira condição a analisar:
\begin{math}
x - 1 = 0
\mapsto x = 0 + 1
\mapsto x = 1
\end{math}
Assim, está provado que "por a + b" que $ x = 1 $!

\subsection{Demonstrações}
Essas operações que acabamos de fazer são denominadas \textbf{demonstrações}. É por meio dessa comunicação que as ideias do matemático são registradas e passadas adiante. Elas são feitas seguindo os princípios da \emph{lógica matemática}.

\subsection{Lógica matemática}
Oriunda da filosofia clássica - lógica essa que trata das formas do pensamento em geral de modo a determinar o que é verdadeiro ou falso -, foi desenvolvida principalmente a partir do século XIX a partir da contribuição de \emph{filósofos} e \emph{matemáticos}.
\par
Para a lógica matemática foi criada uma espécie de \textbf{língua própria} para expressar as ideias dessa lógica, de modo a \emph{uniformizar} a escrita das operações lógicas e suprimir eventuais paradoxos e imprecisões linguísticas.
\subsubsection{Sistema axiomático}
Sistema no qual toda a lógica matemática foi desenvolvida, baseada primariamente nos chamados \textbf{axiomas}, afirmações aceitas como um ponto de partida que têm como fim chegar em um ponto específico.
\par
Esse sistema foi proposto por um matemático de nome "Hilbert" em 1900, que propôs que criassem um sistema axiomático do qual \emph{toda matemática poderia derivar}.
\par
Tal desafio resultou em contribuições fundamentais para áreas específicas da matemática, especialmente na \emph{Álgebra} e na \emph{Teoria dos Conjuntos}, essa que teve muita contribuição do grande Georg \textbf{Cantor}.
\par
Contudo, provou-se que, se um sistema axiomático for rico o suficiente para construir a matemática, ele:
\begin{enumerate}
    \item não será \textbf{completo}, pois há verdades que não podem ser provadas;
    \item não pode provar que não tem contradições em si mesmo.
\end{enumerate}
\par
Mas o que se tem hoje em dia? Bem, os matemáticos utilizam o sistema axiomático denominado \emph{ZFC}, que não vou adentrar aqui pois não nos interessa, é apenas a título de curiosidade.

\subsection{Matemática Discreta}
"Tá, mas onde você quis chegar com toda essa introdução e contextualização? Cadê a matemática discreta?". Bem, aqui estamos: a matemática discreta é uma área da matemática que trata acerca das \textbf{estruturas matemáticas que são finitas e/ou podem ser enumeradas}; um bom exemplo disso são os números naturais. Ela apresenta métodos de resolução diversos e utiliza constantemente do \emph{raciocínio matemático}.

\section{Lógica}

A \emph{lógica matemática} tem como base a \textbf{veracidade de uma proposição}, de modo a determinar se uma afirmação matemática é verdadeira ou falsa; às afirmações verdadeiras damos o nome de \textbf{teoremas}. Por exemplo, você provavelmente sabe que "a razão entre o perímetro e o diâmetro de uma circunferência é sempre $\pi$'". 
Isso é uma verdade inquestionável, independente do tamanho da circunferência. Note que o objeto matemático \textbf{circunferência} é \emph{diferente} do desenho da circunferência, pois, como sabemos, \emph{os objetos matemáticos são abstratos}, não existem na materialidade e são perfeitos.
\par
Na matemática, como comentado anteriormente, exige-se um grande rigor na escrita: não são admitidas como verdadeiras proposições como \emph{"Estude todos os dias", "Ela é linda",} entre outras, pois há lacunas que não foram preenchidas (quem é ela?, o que é ser "linda"?, estudar o quê?).
\subsection{Exemplo}
Mas e quanto à frase \emph{"Todo número inteiro par, maior ou igual a 4, é resultado da soma de dois números primos"}, ela é uma proposição? Sim, visto que, por definição, uma proposição é \emph{uma sentença que é considerada verdadeira ou falsa} (tem um propósito bem objetivo). Mas e quanto ao seu valor lógico, a proposição é verdadeira ou falsa? Podemos analisar, mas façamos as seguintes perguntas:
\begin{itemize}
    \item "O que é um número inteiro?"
    \item "O que é um número par?"
    \item "O que é um número?"
    \item "O que é uma soma?"
    \item ...
\end{itemize}
Note que, só neste exemplo, é possível encontrar mais camadas e camadas de perguntas, mas não é o objetivo desvendar tais questões agora. Admitemos as seguintes definições como verdades:
\begin{enumerate}
    \item \textbf{Definição 1}: \emph{Sendo $a$ e $b$ inteiros, dizemos que $a$ divide $b$ se existir um número $x$ de tal modo que $ a \times x = b $.} Assim, escrevemos $a|b$ indicando que $a$ divide $b$.
    \begin{enumerate}
    \item "6 divide 12?". Com base na definição 1, existe um número $x$ tal que $6 \times x = 12$. Tomemos, portanto, $x = 2$. Agora me diz: 0 é divisor de algum número inteiro? Para isso, suponha, por absurdo, que deve existir um número $z$ tal que $ 0 \times z = b$; no entanto, por definição, o produto de 0 com qualquer número resulta no próprio 0. Então, 0 é divisor de um número inteiro, mas somente dele e de nenhum outro número.
    \item Podemos tirar algumas conclusões dessa definição, como:
    \begin{itemize}
        \item -5 é divisor de 100: $(-5) \times (-20) = 100$;
        \item 21 é divisor de -441: $21 \times (-21) = (-441)$;
        \item 1 e -1 são divisores de quaisquer inteiros: $(-1) \times x = b$.
    \end{itemize}
    \end{enumerate}
    
    \item \textbf{Definição 2}: um número $a$ é par se $2|a$, ou seja, se existe um inteiro $b$ tal que $a = 2 \times b$.
    \begin{itemize}
        \item \emph{Tentativa de} Prova: suponha, por negação, que $a$ não é par, de modo que $ a \ne 2 \times b$ (2 não divide $a$). Porém, por definição, $2|a$, o que nos leva a uma contradição. Se a negação da proposição nos leva a uma contradição, a proposição por si só é verdadeira.
        \footnote{Talvez isso aqui contenha um erro, vou verificar depois.}
    \end{itemize}
    \item \textbf{Definição 3}: um inteiro $a$ é ímpar se existe um inteiro $k$ tal que $a = (2 \times k) + 1$.
    \item \textbf{Definição 4}: um inteiro $p$ é primo se, e somente se, $ p > 1 $ e os únicos divisores de $p$ forem 1 e $p$.
\end{enumerate}

Com base nessas definições, vamos analisar as seguintes proposições e, depois, verificar se a proposição inicial \emph{"Todo número inteiro par, maior ou igual a 4, é resultado da soma de dois números primos"} é verdadeira.

\begin{enumerate}
    \item \emph{"A soma de dois inteiros pares é par"}: com base na Definição 1, tomemos dois inteiros $a$ e $b$, tais que, sendo $x$ e $y$ inteiros quaisquer, $2 \times x = a$ e $2 \times y = b$. Se a soma deles é par, então existe um inteiro $n$ tal que $a + b = 2 \times n$. Como $a = 2 \times x$ e $b = 2 \times y$, temos:
    \begin{equation}
        a + b = 2 \times n
        \Longrightarrow (2 \times x) + (2 \times y) = 2 \times n
        \Longrightarrow 2 \times (x + y) = 2 \times n
    \end{equation}
    Podemos concluir, portanto, que $x + y = n$, e que a proposição inicial é verdadeira e ela é um teorema! 
    \item \emph{"Se o produto dos inteiros $a$ e $b$ for par, um dos dois inteiros é par"}: suponha que os dois inteiros são ímpares, ou seja, tal situação nos deixa com os inteiros $a = (2 \times k) + 1$ e $b = (2 \times j) + 1$, onde $k, j$ são inteiros quaisquer.
    Sendo $n$ o produto $a \times b$, a expressão originada a partir da proposição será, portanto:
        \begin{align*}
        a \times b = n &\implies 
        ((2 \times k) + 1) \times ((2 \times j) + 1) = a \times b \\ &\implies 
        4kj + 2k + 2j + 1 = a \times b \\ &\implies
        2 \times (2kj + k + j) + 1 = a \times b \\ &\implies
        2 \times c + 1 = a \times b
        \end{align*}
    Reduzindo a expressão ao inteiro $c$, temos que o produto $a \times b$ é ímpar, o que é uma contradição, já que no ínício definimos tal produto como par! Então, a contraposição da proposição é falsa, o que torna a proposição verdade!
    \footnote{Acredito que a demonstração não está errada não, mesmo sendo eu quem fiz; de qualquer forma, a proposição é mesmo verdadeira.}
    \item \emph{"Se $a|b$ e $b|a$, então $a = b$"}: essa proposição é falsa, vejamos um contraexemplo. Sejam os inteiros $a = 1$ e $b = -1$; temos que $ 1 \times (-1) = (-1)$ e $(-1) \times (-1) = 1$. No entanto, $a \ne b$!
    \item \emph{"Todo número inteiro par, maior ou igual a 4, é resultado da soma de dois números primos"}: essa afirmação é uma \textbf{Conjectura}; não se sabe de nenhum contraexemplo até o presente momento, mas também é algo que não se pode provar de maneira abstrata pelas proposições acima nem outras.
\end{enumerate}
Pode ter sido muita informação por agora, mas isso é algo que a gente se acostuma com o tempo. De toda forma, vamos dar uma olhada na "linguagem da lógica", que nos permitiu fazer todas essas operações.

\section{Linguagem e Lógica}
Como notado e utilizado, a linguagem matemática deve ser muito precisa e todos os seus termos, declarados. A partir daqui, devemos observar com mais aprofundamento os aspectos da linguagem matemática, mais precisamente como sua lógica é construída.
\subsection{Princípios básicos da proposição}
Uma proposição deve obedecer às seguintes restrições:
\begin{enumerate}
    \item \textbf{Princípio da identidade}: uma proposição verdadeira é verdadeira, e uma falsa é falsa;
    \item \textbf{Princípio da não-contradição}: nenhuma proposição pode ser verdadeira e falsa simultaneamente; 
    \item \textbf{Princípio do terceiro excluído}: uma proposição só pode ser verdadeira \textbf{ou} falsa, não havendo uma terceira possibilidade.
\end{enumerate}
Por mais que pareça algo óbvio, é fundamental que tais princípios sejam definidos antes de proposições serem operadas para evitar contradições e sabermos as "regras" do jogo.
\subsection{Tabela-verdade e Conjunções}
Na lógica, a estrutura da \emph{tabela-verdade} serve para listar todos as possibilidades de uma interação entre proposições. Tomando as proposições A e B, é possível listar seus valores da seguinte forma: \\
\begin{align*}
\begin{tabular}{c|c}
A & B \\
\hline
V & V \\
V & F \\
F & V \\
F & F \\
\end{tabular}
\end{align*}

Agora, temos o total de possibilidades da interação entre as proposições. No entanto, elas ainda não estão se relacionando; para que isso aconteça, devemos utilizar das \emph{conjunções}, de modo a manipular esses valores e gerar um valor final. Vamos analisar cada uma das conjunções:

\subsubsection{Conjunção \emph{AND}}
A conjunção \emph{end}, escrita como $\land$, gera somente um valor verdadeiro \emph{se as duas proposições forem verdadeiras}.
Tabela-verdade:
\begin{align*}
    \begin{tabular}{cc|c}
         A & B & A $\land$ B  \\
        \hline
         V & V & V \\
         V & F & F \\
         F & V & F \\
         F & F & F \\
    \end{tabular}
\end{align*}

\subsubsection{Conjunção \emph{OR}}
A conjunção \emph{or}, escrita como $\lor$, gera um valor verdadeiro quando \emph{ao menos uma proposição é verdadeira}.
Tabela-verdade:
\begin{align*}
    \begin{tabular}{cc|c}
         A & B & A $\lor$ B \\
         \hline
         V & V & V \\
         V & F & V \\
         F & V & V \\
         F & F & F \\
    \end{tabular}
\end{align*}

\subsubsection{Conjunção \emph{XOR}}
A conjunção \emph{xor}, escrita como $\vee$, gera um valor verdadeiro quando \emph{uma proposição é verdadeira ou outra}.
Tabela-verdade:
\begin{align*}
    \begin{tabular}{cc|c}
         A & B & A $\vee$ B \\
         \hline
         V & V & F \\
         V & F & V \\
         F & V & V \\
         F & F & F \\
    \end{tabular}
\end{align*}

\subsubsection{Conjunção \emph{IMPLIES}}
A conjunção \emph{implies}, escrita como $\implies$, indica as relações:
\begin{itemize}
    \item \emph{Se A, então B};
    \item \emph{A somente, se B};
    \item \emph{A é condição suficiente para B};
    \item \emph{B segue de A};
    \item \emph{B é condição necessária para A}.
\end{itemize}
Ou seja, ela gera um valor positivo em todos os casos exceto quando a proposição $B$ for falsa, pois $B$ é consequência de $A$. 
\textbf{OBS.:}: Se $A \implies B$, então $\lnot B \implies \lnot A$.
Tabela-verdade:
\begin{align*}
    \begin{tabular}{cc|c}
         A & B & A $\implies$ B  \\
         \hline
         V & V & V \\
         V & F & F \\
         F & V & V \\
         F & F & V \\
    \end{tabular}
\end{align*}

\subsubsection{Conjunção \emph{IF AND ONLY IF}}
A conjunção de equivalência \emph{if and only if}, escrita como $\iff$, indica a relação \emph{"A se, e somente se B"} e \emph{"A é condição necessária e suficiente para B"}, de modo que gera um valor positivo somente quando as duas têm o mesmo valor.
Tabela-verdade:
\begin{align*}
    \begin{tabular}{cc|c}
         A & B & A $\iff$ B  \\
         \hline
         V & V & V \\
         V & F & F \\
         F & V & F \\
         F & F & V \\
    \end{tabular}
\end{align*}

\subsubsection{Conjunção \emph{NOT}}
Ao contrário das anteriores, a conjunção \emph{not}, escrita como $\lnot$, \textbf{inverte o valor de uma proposição}. Ou seja, a tabela-verdade seria:
\begin{align*}
    \begin{tabular}{c|c}
        A & $\lnot$ A  \\
        \hline
        V & F \\
        F & V \\
    \end{tabular}
\end{align*}

\section{To Be Continued...}

\end{document}
